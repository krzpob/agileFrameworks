\documentclass[epic,eepic,aspectratio=169,12pt]{beamer}
\usepackage[polish]{babel}
\usepackage[T1]{polski}
\usepackage[utf8]{inputenc}
\usepackage[T1]{fontenc}
\usepackage{color}
\usepackage{picture}
\usepackage{graphicx}
\usepackage{minibox}
\usepackage{csquotes}
%\usetheme{Copenhagen}
\usecolortheme{crane}

\usepackage[]{csquotes}
\usepackage[sorting=nty,isbn=true,backend=biber]{biblatex}

\title{Nie tylko SCRUM jest zwinny}

\author{Krzysztof Pobożan}
\subtitle{Czyli o innych zwinhych technikach}

\begin{document}
\begin{frame}
	\maketitle
\end{frame}
	\begin{frame}{Agenda}
		\tableofcontents
	\end{frame}
\section{Kanban}
\begin{frame}{Kanban - definicja}
	Środek do tworzenia, zarządzania i poprawy przepływo zadań dla znanej pracy.
	Pozwala organizacją w sposób płynny zmienić obecny system pracy. 
	Mogą to zrobić poprzez tablicę kanbanową.
	
	Jedynym ograniczneniem jest WIP - Work In Progres ilosć zadań na raz na danym etapie pracy.
	
	Pozwala skupić się na kończeniu pracy zamiast na zaczynaniu.
	
\end{frame}
\begin{frame}{Kanban - gdzie używać}
	\begin{itemize}
		\item Sposób wykonania zadań jest znany.
		\item Zadania napływają w sposób nie kontrolowany.
		\item Chcemy wdrażać pracę jak tylko jest gotowa, bez czekania na inne zadania.
	\end{itemize}
\end{frame}
\begin{frame}{Kanban - wartości}
	Przezroczystość, Równowaga, Współpraca, Skupienie na kliencie, Przepływ, Przywództwo, Zrozumienie, Szacunek
\end{frame}
\section{Scrumban}
\begin{frame}{Scrumban - definicja}
	
\end{frame}
\section{XP}

\end{document}